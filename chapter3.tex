% !TeX program = pdflatex
% !TeX root=./main.tex
% Edit by:XAjzh and sikouhjw

\chapter{多维随机变量及其分布}\label{cha:3}
在有些随机现象中, 对每个样本点 $\omega$ 只用一个随机变量去描述是不够的, 
譬如要研究儿童的生长发育情况, 仅研究儿童的身高 $X(\omega)$ 或仅研究其体重 $Y(\omega)$都是片面的, 
有必要把 $X(\omega)$ 和 $Y(\omega)$ 作为一个整体来考虑, 讨论它们总体变化的统计规律性, 
进一步可以讨论 $X(\omega)$ 与 $Y(\omega)$ 之间的关系, 在有些随机现象中, 甚至要同时研究二个以上随机变量. 

如何来研究多维随机变量的统计规律性呢, 仿一维随机变量, 我们先研究联合分布函数, 
然后研究离散随机变量的联合分布列、连续随机变量的联合密度函数. 

\section{多维随机变量及其联合分布}\label{sec:3.1}

\subsection{多维随机变量}\label{ssec:3.1.1}
下面我们先给出 $n$ 维随机变量的定义. 
\begin{definition}{随机变量}{3.1.1}
	如果 $X_1(\omega),X_2(\omega),\ldots,X_n(\omega)$ 是定义在同一
	样本空间 $\Omega=\left\{\omega\right\}$ 上的 $n$ 个随机变量, 则称
	\[
	X(\omega)=(X_1(\omega),X_2(\omega),\ldots,X_n(\omega))
	\]
	为 $n$ 维(或 $n$ 元)\textbf{随机变量}\index{S!随机变量}或\textbf{随机向量}\index{S!随机向量}. 
\end{definition}
注意, 多维随机变量的关键是定义在同一样本空间上, 对于不同样本空间上的两个随机变量, 我们只能在
乘积空间 $\Omega_1\times \Omega_2=\left\{(\omega_1,\omega_2);\omega_1 \in \Omega_1,\omega_2 \in \Omega_2 \right\}$ 上讨论, 
 这要用到更多的工具, 本章将不涉及这类问题. 
      
 在实际问题中, 多维随机变量的情况是经常会遇到的譬如
  \begin{itemize}
  	\item 在研究四岁至六岁儿童的生长发育情况时, 我们感兴趣于每个儿童 (样本点 $\omega$) 的身高 $X_1(\omega)$ 
  	和体重 $X_2(\omega)$ . 这里 $(X_1(\omega),X_2(\omega))$ 是一个二维随机变量. 
  	\item 在研究每个家庭的支出情况时, 我们感兴趣于每个家庭 (样本点 $\omega$) 的衣食住行四个方面, 
  	若用 $X_1(\omega),X_2(\omega),X_3(\omega),X_4(\omega)$ 分别表示衣食住行的花费占其家庭总收人的百分比, 
  	则 $(X_1(\omega),X_2(\omega),X_3(\omega),X_4(\omega))$ 就是一个四维随机变量. 
  \end{itemize}

  \subsection{联合分布函数}\label{ssec:3.1.2}
  \begin{definition}{联合分布函数}{3.1.2}
  	对任意的 $n$ 个实数 $x_1,x_2,\ldots,x_n ,$ 则 $n$ 个事件 $\{X_1\leq x_1\},\{X_2 \leq x_2\},\ldots,\{X_n\leq x_n\}$ 同时发生的概率
    \begin{equation}
    	F\left(x_{1}, x_{2}, \cdots, x_{n}\right)=P\left(X_{1} \leq x_{1}, X_{2} \leq x_{2}, \ldots, X_{n} \leq x_{n}\right)\label{eq:3.1.1}
    \end{equation}
	称为 $n$ 维随机变量 $(X_1,X_2,\ldots,X_n)$ 的\textbf{联合分布函数}\index{L!联合分布函数}. 
  \end{definition}
   本章主要研究二维随机变量, 二维以上的情况可类似进行. 

   在二维随机变量 $(X,Y)$ 场合, 联合分布函数 $F(x,y)=P(X\leq x,Y\leq y)$ 是事件 $\{X\leq x\}$ 与 $\{Y \leq y\}$ 同时发生(交)的概率. 
   如果将二维随机变量 $(X,Y)$ 看成是平面上随机点的坐标, 那么联合分布函数 $F(x,y)$ 在 $(x,y)$ 处的函数值就是随机点 $(X, Y)$ 落在
   以 $(x,y)$ 为右上角的无穷矩形内的概率, 见图~\ref{fig:3.1.1} .
   \begin{figure}[htbp]
   	\centering
   	\includegraphics[width=0.4\textwidth]{fig3-1-1.pdf}
   	\caption{联合分布函数示意图}\label{fig:3.1.1}
   \end{figure}
   \begin{theorem}{}{3.1.1}
   	任一二维联合分布函数 $F(x,y)$ 必具有如下四条基本性质: 
    \begin{enumerate}
    	\item \textbf{单调性}\quad $F(x,y)$ 分别对 $x$ 或 $y$ 是单调不减的,即
    	\begin{itemize}
    		\item 当 $x_1<x_2$ 时, 有 $F(x_1,y)\leq F(x_2,y)$.
    		\item 当 $y_1<y_2$ 时, 有 $F(x,y_1)\leq F(x,y_2)$. 
    	\end{itemize}
    	\item \textbf{有界性}\quad 对任意的 $x$ 和 $y$, 有 $0\leq F(x,y) \leq 1$, 且
    	\begin{align*}
    		&F(-\infty,y)=\lim_{x\to -\infty}F(x,y)=0,	&\hspace*{3cm} \\
    		&F(x,-\infty)=\lim_{y\to -\infty}F(x,y)=0,	&\\
    		&F(+\infty,+\infty)=\lim_{x,y\to +\infty}(x,y)=1.&
    	\end{align*}
    	\item \textbf{右连续性}\quad  对每个变量都是右连续的,即
    		\begin{align*} 
    			F(x+0, y) &=F(x, y), &\hspace*{3cm} \\ 
    			F(x, y+0) &=F(x, y). &
    		\end{align*}
    	\item \textbf{非负性}\quad 对任意的 $a<b,c<d$ 有
    		\begin{equation*}
    		 	P(a<X \leq b, c<Y \leq d)= F(b, d)-F(a, d)-F(b, c)+ F(a, c) \geq 0 .
    		\end{equation*}
    \end{enumerate}
   \end{theorem}
   \begin{proof}
    	\begin{enumerate}
    		\item  因为当 $x_1<x_2$ 时, 有 $\{X_1\leq x_1\} \subset \{X_2 \leq x_2\}$, 所以对任意给定的 $y$ 有
    		\[
    			\left\{X \leq x_{1}, Y \leq y\right\} \subseteq \{ X \leq x_{2}, Y \leq y \},
    		\]
    		由此可得
    		\[
    			F\left(x_{1}, y\right)=P\left(X \leq x_{1}, Y \leq y\right) \leq 
    			P\left(X \leq x_{2}, Y \leq y\right)=F\left(x_{2}, y\right),
    		\]
    		即 $F(x,y)$ 关于 $x$ 是单调不减的, 同理可证 $F(x,y)$ 关于 $y$ 是单调不减的.
    		\item 由概率的性质可知 $0\leq F(x,y) \leq 1$. 又因为对任意的正整数 $n$ 有
    		\begin{align*}
    			&\lim_{x\to -\infty}\{X\leq x\}=\lim_{n\to +\infty}\bigcap_{m=1}^n \{X\leq -m\}=\varnothing,	\\
    			&\lim_{x\to +\infty}\{X\leq x\}=\lim_{n\to +\infty}\bigcup_{m=1}^n\{X\leq m\}=\Omega,
    		\end{align*}
    		对 $Y \leq y$ 也类似可得. 再由概率的连续性, 就可得
    		\[
    		 	F(-\infty, y)=F(x,-\infty)=0; \quad F(+\infty,+\infty)=1.
    		\]
    		\item 固定 $y$, 仿一维分布函数右连续的证明, 就可得知 $F(x, y)$ 关于 $x$ 是右连续的. 同样固定 $x$ 可
    		证得 $F(x,y)$ 关于 $y$ 是右连续的.
        	\item 只需证
        	\[
        	 	P(a<X \leq b, c<Y \leq d)=F(b, d)-F(a, d)-F(b, c)+F(a, c).
        	\]
        	为此记 ( 见图~\ref{fig:3.1.2} )
        	\[
        	 	A=\{X \leqslant a\}, \quad B=\{ X \leqslant b \}, \quad C=\{Y \leqslant c\}, \quad D=\{Y \leqslant d\},
        	\]
        	考虑到
        	\[
        	 	\{ a<X \leqslant b \}=B-A=B \cap \overline{A}, \quad\{c<Y \leqslant d\}=D-C=D \cap \overline{C},
        	\]
        	且 $A \subset B, C\subset D,$ 由此可得
        	\begin{align*}
        		 0 & \leqslant P(a<X \leqslant b, c<Y \leqslant d) \\ 
        		 &=P(B \cap \overline{A} \cap D \cap \overline{C}) \\ 
        		 &=P(B D-(A \cup C)) \\ 
        		 &=P(B D)-P(A B D \cup B C D) \\ 
        		 &=P(B D)-P(A D \cup B C) \\ 
        		 &=P(B D)-P(A D)-P(B C)+P(A B C D) \\ 
        		 &=P(B D)-P(A D)-P(B C)+P(A C ) \\ 
        		 &=P(B D)-P(A D)-P(B C)+P(A B C D) \\ 
        		 &=F(b, d)-F(a, d)-F(b, c)+F(a, c). 
        	\end{align*}
    	\end{enumerate}
    \end{proof}
   \begin{figure}[htbp]
   	\centering
   	\includegraphics[width=0.4\textwidth]{fig3-1-2.pdf}
   	\caption{二维随机变量 $(X,Y)$ 落在矩形中的情况}\label{fig:3.1.2}
   \end{figure}
   还可证明,具有上述四条性质的二元函数 $F(x,y)$ 一定是某个二维随机变量的分布函数.

   任一二维分布函数 $F(x,y)$ 必具有上述四条性质, 其中性质 \textit{4} 是二维场合特有的, 也是合理的. 但性质 \textit{4} 不能
   由前三条性质推出, 必须单独列出, 且仅满足前三条性质是不够的, 因为存在这样的二元函数 $G(x,y)$ 满足
   以上性质 \textit{1,2,3}, 但它不满足性质 \textit{4}, 见下面例子.

   \begin{example}\label{exam:3.1.1}
   	二元函数
   	\[
   	 	G(x,y)=\begin{cases}
   	 		0,	& x+y<0;\\
   	 		1,	& x+y\geq 0 \\
   	 	\end{cases}
   	\]
   	满足二维分布函数的性质 \textit{1,2,3}, 但它不满足性质 \textit{4}.

    这从 $G(x,y)$ 的定义可看出: 若用 $x+y=0$ 将平面 $xOy$ 一分为二, 则
 
    $G(x,y)$ 在右上半平面 $(x+y\geq 0)$ 取值为1,
   
    $G(x,y)$ 在左下半平面 $(x+y\geq 0)$ 取值为0,

    $G(x,y)$ 具有非降性、有界性和右连续性, 但在正方形区域 $\{(x,y); -1\leq x\leq 1,-1\leq y\leq 1\}$的四个顶点上, 右上
    三个顶点位于右上半闭平面, 只有左下顶点 $(-1,-1)$ 位于左下半开平面, 故有
    \[
     	G(1,1)-G(1,-1)-G(-1,1)+G(-1,-1)=1-1-1+0=-1<0,
    \]
	所以 $G(x,y)$ 不满足性质 \textit{4}, 故 $G(x,y)$ 不能成为某二维随机变量的分布函数.
   \end{example}

   \subsection{联合分布列}\label{ssec:3.1.3}

   \begin{definition}{联合分布列}{3.1.3}
		如果二维随机变量 $(X,Y)$ 只取有限个或可列个数对 $(x_i,y_j)$, 则称 $(X,Y)$ 为二维离散随机变量, 称
    \begin{equation}\label{eq:3.1.2}
     	p_{i j}=P\left(X=x_{i}, Y=y_{j}\right), \quad i, j=1,2, \ldots
    \end{equation}
   	为 $(X,Y)$ 的\textbf{联合分布列}\index{L!联合分布列}, 也可如下用表格形式记联合分布列.
   \end{definition}
   \begin{center}
   		\begin{tabularx}{0.8\textwidth}{ZZZZZZ}
   		\toprule
   		 & \multicolumn{5}{c}{$Y$} \\
   		 \cmidrule{2-6} 
   		 $X$	&	$y_1$&	$y_2$&	\ldots&	$y_j$& \ldots	\\
   		 \midrule 
   		 $x_1$& $p_{11}$ & $p_{12}$ & \ldots & $p_{1j}$ & \ldots \\
   		 \midrule 
   		 $x_2$& $p_{21}$ & $p_{22}$ & \ldots & $p_{2j}$ & \ldots \\
   		 \midrule 
   		 $\vdots$& $\vdots$ & $\vdots$ & $\ddots$ & $\vdots$ & $\ddots$ \\
   		 \midrule 
   		$x_i$ & $p_{i1}$ & $p_{i2}$ & \ldots & $p_{ij}$  & \ldots \\
   		\midrule 
   		 $\vdots$& $\vdots$ & $\vdots$ & $\ddots$ & $\vdots$ & $\ddots$ \\
   		 \bottomrule
   		\end{tabularx} 
   \end{center}
   联合分布函数的基本性质:
   \begin{enumerate}
   	\item 非负性: $p_{ij}\geq 0$;
   	\item 正则性: $\sum_{i=1}^{+\infty}\sum_{j=1}^{+\infty}p_{ij}=1$.
   \end{enumerate}
   求二维离散随机变量的联合分布列, 关键是写出二维随机变量的可能取的数对及其发生的概率.

   \begin{example}\label{exam:3.1.2}
   		从 1,2,3,4 中任取一数记为 $X$, 再从 $1,\ldots,X$ 中任取一数记为 $Y$. 求 $(X,Y)$ 的联合分布列及 $P(X=Y)$.     
   \end{example}
   \begin{solution}
   		$(X,Y)$ 为二维离散随机变量, 其中 $X$ 的分布列为
         \[  
         	 P(X=i)=1/4,i=1,2,3,4.
         \]
		$Y$ 的可能取值也是 $1,2,3,4$, 若记 $j$ 为 $Y$ 的取值,

		 则当 $j>i$ 时,有 $P(X=i,Y=j)=P(B)=0$.
  
  		当 $1\leq j \leq i \leq 4$ 时,由乘法公式
  		\[
  		 	P(X=i, Y=j)=P(X=i) P(Y=j | X=i)=\frac{1}{4} \times \frac{1}{i}.
  		\]
		所以得 $(X,Y)$ 的联合分布列为
		\begin{center}
			\begin{tabularx}{0.8\textwidth}{ZZZZZ}
			\toprule
			 & \multicolumn{4}{c}{$Y$} 		 \\ 
			 \cmidrule{2-5}
			 $X$&  	1	&  2	&  3&  		4\\ 
			 \midrule
			 1 	&  1/4	&  0	&  0&  		0\\
			 \midrule
			 2	&  1/8	& 1/8 	&  0&  		0\\ 
			 \midrule
			 3	& 1/12	&  1/12	&  1/12&	0\\ 
			 \midrule
			 4&  1/16	&  1/16	&  1/16&  1/16\\ 
			 \bottomrule
			\end{tabularx} 
		\end{center}
		由此可算得事件 $\{X=Y\}$ 的概率为
		\[
		 	P(X=Y)=p_{11}+p_{22}+p_{33}+p_{44}=\frac{1}{4}+\frac{1}{8}+\frac{1}{12}+\frac{1}{12}=\frac{25}{48}=0.5208
		\]
   \end{solution}
   \subsection{联合密度函数}\label{ssec:3.1.4}
   \begin{definition}{联合密度函数}{3.1.4}
   	如果存在二元非负函数 $p(x,y)$, 使得二维随机变量 $(X,Y)$ 的分布函数 $F(x,y)$ 可表示为
   	\begin{equation}\label{eq:3.1.3}
   	 	F(x, y)=\int_{-\infty}^{x} \int_{-\infty}^{y} p(u, v) \dd v \dd u
   	\end{equation}
   	则称 $(X,Y)$ 为二维连续随机变量, 称 $p(u,v)$ 为 $(X,Y)$ 的\textbf{联合密度函数}\index{L!联合密度函数}.
   \end{definition}
   在 $F(x,y)$ 偏导数存在的点上有
   \[
    	p(x, y)=\frac{\partial^{2}}{\partial x \partial y} F(x, y).
   \]
   \textbf{联合密度函数的基本性质}:
   \begin{enumerate}
   	\item \textbf{非负性}: $p(u,v)\geq 0$ 
   	\item \textbf{正则性}: $\int_{-\infty}^{+\infty} \int_{-\infty}^{+\infty} p(u, v) \dd v \dd u=1$
   \end{enumerate}
   给出联合密度函数 $p(x,y)$, 就可以求有关事件的概率了. 若 $G$ 为平面上的一个区域, 
   则事件 $\{(X,Y) \in G\}$ 的概率可表示为在 $G$ 上对 $p(x,y)$的二重积分:
   \begin{equation}\label{eq:3.1.4}
    	P((X, Y) \in G)=\iint_{G} p(x, y) \dd x \dd y.
   \end{equation}
   在具体使用上式时, 要注意积分范围是 $p(x,y)$ 的非零区域与 $G$ 的交集部分, 然后设法化成累次积分, 最后计算出结果.
   \begin{example}\label{exam:3.1.3}
   		设 $(X,Y)$ 联合密度函数为
   		\[
   		p(x, y)=\left\{
   		\begin{array}{cc}
   		6 \ee^{-2 x-3 y}, & x>0, y>0; \\
   		0, & \text{其他} .
   		\end{array}\right.
   		\]
   		试求: $(1)\; P(X<1,Y>1); \; (2)\; P(X>Y)$.
   \end{example}
   \begin{solution}
  		\begin{enumerate}
  		\item 积分区域见图~\ref{fig:3.1.3a} 中的阴影部分, 
  		\begin{align*} 
  		 P(X<1, Y>1) &=\int_{1}^{+\infty} \int_{0}^{1} 6 \ee^{-2 x-3 y} \dd x \dd y \\
  		 &=6 \int_{0}^{1} \ee^{-2 x} \dd x \int_{1}^{+\infty} \ee^{-3 x} \dd y \\ 
  		 &=\left(1-\ee^{-2}\right) \ee^{-3}=0.0430. 
  		\end{align*}
  		\item 积分区域见图~\ref{fig:3.1.3b} 中的阴影部分, 从而容易写出累次积分.
  		\begin{align*} 
  		 P(X>Y) &=\int_{0}^{+\infty} \int_{0}^{x} 6 \ee^{-2 x} \ee^{-3 y} \dd y \dd x=
  		 \int_{0}^{+\infty} 2 \ee^{-2 x}\left(1-\ee^{-3 x}\right) \dd x \\ 
  		 &=\left[-\ee^{-2 x}+\frac{1}{5} \ee^{-5 x}\right]_{0}^{+\infty}=1-\frac{1}{5}=\frac{4}{5} .
  		\end{align*}
  		\end{enumerate}             
   \end{solution}
   \begin{figure}[htbp]
   \centering
   \subfloat[$\{x<1,y>1\}$ 区域 $D_1$]{\label{fig:3.1.3a}
   \includegraphics[width=0.4\textwidth]{fig3-1-3a.pdf}
   }\qquad
   \subfloat[$\{x>y\}$ 区域 $D_2$]{\label{fig:3.1.3b}
   \includegraphics[width=0.4\textwidth]{fig3-1-3b.pdf}
   }
   \caption{$p(x,y)$ 的非零区域与有关事件的交集部分}\label{fig:3.1.3}
   \end{figure}
   
  \subsection{常用多维分布}\label{ssec:3.1.5}
  下面介绍一些多维随机变量的常用分布. 
   
  \textbf{一、多项分布 }
   
  多项分布是重要的多维离散分布, 它是二项分布的推广.
  
  进行 $n$ 次独立重复试验, 如果每次试验有 $r$ 个可能结果: $A_1, A_2$,\ldots,$A_r$, 且每次试验中 $A_i$ 发生的
  概率为 $p_{i}=P\left(A_{i}\right), i=1,2, \ldots, r . p_{1}+p_{2}+\cdots+p_{r}=1$.
  记 $X_i$ 为 $n$ 次独立重复试验中 $A_i$ 出现的次数, $i=1,2,\ldots,r$. 则 $(X_1,X_2,\ldots,X_r)$ 
  取值 $(n_1,n_2,\ldots,n_r)$ 的概率, 即 $A_1$ 出现 $n_1$ 次, $A_2$ 出现$n_2$ 次\ldots\ldots $A_r$ 出现 $n_r$ 次的概率为
  \begin{equation}
	  P\left(X_{1}=n_{1}, X_{2}=n_{2}, \ldots, X_{r}=n_{r}\right)=\frac{n !}{n_{1} ! n_{2} ! \cdots n_{r} !} p_{1}^{n_{1}} p_{2}^{n_{2}} \ldots p_{r}^{n_r},
  \end{equation}
  其中 $n=n_1+n_2+\cdots+n_r$.
  
  这个联合分布列称为 $r$ 项分布, 又称多项分布, 记为 $M(n,p_1,p_2,\ldots,p_r)$. 这个概率
  是多项式 $(p_1+p_2+\cdots+p_r)^n$ 展开式中的一项, 故其和为 1. 当 $r=2$ 时, 即为二项分布.
  \begin{example}\label{exam:3.1.4}
    一批产品共有 100 件, 其中一等品 60 件、二等品 30 件、三等品 10 件. 
    从这批产品中有放回地任取3件, 以 $X$ 和 $Y$ 分别表示取出的 3 件产品中一等品、二等品的件数, 求二维随机变量 $(X,Y)$ 的联合分布列.
  \end{example}
  \begin{solution}
    因为 $X$ 和 $Y$ 的可能取值都是 0,1,2,3, 所以记 $(X,Y)$ 的联合分布列为
    \begin{center}
      \begin{tabularx}{0.8\textwidth}{ZZZZZ}
        \toprule
          & \multicolumn{4}{c}{$Y$} \\
        \cmidrule{2-5}
        $X$ & 0&  1&  2&  3 \\
        \midrule
        0&  $p_{00}$& $p_{01}$& $p_{02}$& $p_{03}$\\
        \midrule
        1&  $p_{10}$& $p_{11}$& $p_{12}$& $p_{13}$\\
        \midrule
        2&  $p_{20}$& $p_{21}$& $p_{22}$& $p_{23}$\\
        \midrule
        3&  $p_{30}$& $p_{31}$& $p_{32}$& $p_{33}$\\
        \bottomrule
      \end{tabularx}
    \end{center}
    当 $i+j>3$ 时, 有 $p_{ij}=0$, 即
    \[
      p_{13}=p_{22}=p_{23}=p_{31}=p_{32}=p_{33}=0.
    \]
    而当 $i+j\leq 3$ 时, 事件 $\{X=i,Y=j\}$ 表示: 取出的 3 件产品中有 $i$ 件一等品、$j$ 件二等品、$3-i-j$ 件三等品的件数, 
    所以有放回地抽取时, 对 $i+j\leq 3$, 有
    \[
      p_{i j}=\frac{3 !}{i ! j !(3-i-j) !}\left(\frac{6}{10}\right)^{i}\left(\frac{3}{10}\right)^{j}\left(\frac{1}{10}\right)^{3-i-j}.
    \]
    由以上公式, 就可具体算出 $(X,Y)$ 的联合分布列为
    \begin{center}
      \begin{tabularx}{0.8\textwidth}{ZZZZZ}
        \toprule
          & \multicolumn{4}{c}{$Y$} \\
        \cmidrule{2-5}
        $X$ & 0&  1&  2&  3 \\
        \midrule
        0&  0.001   & 0.009   & 0.027   & 0.027   \\
        \midrule
        1&  0.018   & 0.108   & 0.162   & 0       \\
        \midrule
        2&  0.108   & 0.324   & 0       & 0       \\
        \midrule
        3&  0.216   & 0       & 0       & 0       \\
        \bottomrule
      \end{tabularx}
    \end{center}
  \end{solution}
  有此联合分布列, 就可计算有关事件的概率, 譬如
  \begin{align*}
    P(X \leq 1, Y \leq 1)&=0.001+0.009+0.018+0.108=0.136. \\
    P(X=0)=\sum_{j=0}^{3} P(X&=0, Y=j)=0.001+0.009+0.027+0.027=0.064.
  \end{align*}
  %此例是第二章~\ref{sec:2.4} 节中的二项分布的推广, 差别在于:~\ref{sec:2.4} 节中讨论的是从 " 合格品 "、" 不合格品" 两种情况中抽取, 而在
  %此是从一等品、二等品和三等品三种情况中抽取. 这里我们称它为三项分布, 它是一种特殊的多项分布.
  此例是第二章 2.4 节中的二项分布的推广, 差别在于: 2.4 中讨论的是从 " 合格品 "、" 不合格品" 两种情况中抽取, 而在此是从一等品、二等品和
  三等品三种情况中抽取. 这里我们称它为三项分布, 它是一种特殊的多项分布.

  \textbf{二、多维超几何分布 }

  以下给出多维超几何分布的描述: 袋中有 $N$ 只球, 其中有 $N_i$ 只 $i$ 号球, $i=1,2,\ldots,r$,
 记 $N=N_1+N_2+\cdots+N_r$. 从中任意取出 $n$ 只, 若记 $X_i$ 为取出的 $n$ 只球中 $i$ 号球的个数, $i=1,2,\ldots,r$, 则
 \begin{equation}\label{eq:3.1.6}
  	P(X_1=n_1,X_2=n_2,\ldots,X_r=n_r)=\frac	{\binom{N_1}{n_1}\binom{N_2}{n_2}\cdots\binom{N_r}{n_r}}{\binom{N}{n}},
 \end{equation}
 其中 $n_1+n_2+\cdots+n_r=n$.
 \begin{example}\label{exam:3.1.5}
 将例~\ref{exam:3.1.4} 改成不放回抽样, 即从这批产品中不放回地任取 3 件, 记 $x$ 和 $Y$ 分别表示 3 件产品中一等品和二等品的件数, 
 求二维随机变 $(X,Y)$ 的联合分布列. 
 \end{example}
 \begin{solution}
记 $i$ 与 $j$ 分别为 $X$ 与 $Y$ 的取值, 此时对 $i+j>3$, 有 $p_{ij}=0$, 即
\[
	p_{13}=p_{22}=p_{23}=p_{31}=p_{32}=p_{33}=0.
\]
对于 $i+j\leq 3$, 有
\[
	p_{ij}=\frac{\binom{60}{i}\binom{30}{j}\binom{10}{3-i-j}}{\binom{100}{3}}.
\]
由此可得 $(X,Y)$ 的联合分布列,譬如
\[
P(X=1, Y=2)=\frac{\binom{60}{1}\binom{30}{2}}{\binom{100}{3}}
=\frac{60 \times 30 \times 29 / 2}{100 \times 99 \times 98 / 6}=\frac{89}{539}=0.1614
\]
其他各概率都类似求出, 最后得如下联合分布列
\begin{center}
\begin{tabularx}{0.8\textwidth}{ZZZZZ}
\toprule
&\multicolumn{4}{c}{$Y$}	\\
\cmidrule{2-5}
$X$	&	0&	1&	2&	3		\\
\midrule
0	&	0.0007&	0.0083&	0.0269&	0.0251	\\
\midrule
1	&	0.0167&	0.1113&	0.1614&	0		\\
\midrule
2	&	0.1095&	0.3284&	0&		0		\\
\midrule
3	&	0.2116&	0&		0&		0		\\			
\bottomrule
\end{tabularx}
\end{center}
有此联合分布列,就可计算有关事件的概率,譬如
\[
\begin{gathered}
P(X \leq 1, Y \leq 1)=0.0007+0.0167+0.0083+0.1113=0.1370 \\
P(X=0)=\sum_{j=0}^{3} P(X=0, Y=j)=0.0610
\end{gathered}
\]
\end{solution}
 此例是超几何分布的推广, 差别在于: 2.4 中讨论的是从“合格品”、“不合格品”两种情况中抽取, 而在此是从一等品、二等品
 和三等品三种情况中抽取. 这里我们称它为三维超几何分布, 它是一种特殊的多维超几何分布. 

 \textbf{三、多维均匀分布}

设 $D$ 为 $R^n$ 中的一个有界区域, 其度量(平面上为面积, 空间为体积等)为 $S_D$, 如果多维随机变量 $(X_1,X_2,\ldots,X_n)$ 的
联合密度函数为
 \begin{equation}\label{eq:3.1.7}
 p(x_{1}, x_{2}, \cdots, x_{n}=\begin{cases}
\frac{1}{S_{D}}, & (x_{1}, x_{2}, \ldots, x_{n} \in D \\
0,&	\text{其他} \\
\end{cases}.
 \end{equation}
 则称 $(X_1,X_2,\ldots,X_n)$ 服从 $D$ 上的\textbf{多维均匀分布}\index{D!多维均匀分布}, 记为 $(X_1,X_2,\ldots,X_n)\sim U(D)$.

二维均匀分布所描述的随机现象就是向平面区域 $D$ 中随机投点, 如果该点坐标 $(X,Y)$ 落在 $D$ 的
子区域 $G$ 中的概率只与 $G$ 的面积有关, 而与 $G$ 的位置无关, 则由第一章知这是几何概率. 现在由二维均匀分布来描述,则
\[
P((X,Y)\in G)=\iint_G p(x,y) \dd x \dd y=\\int_G \frac{1}{S_D} \dd x \dd y =\frac{G\text{的面积}}{D\text{的面积}}.
\]
这正是几何概率的计算公式. 
\begin{example}\label{exam:3.1.6}
设 $D$ 为平面上以原点为圆心、以 $r$ 为半径的圆, $(X,Y)$ 服从 $D$ 上的二维均匀分布, 其密度函数为
\[
p(x, y)=\begin{cases}\frac{1}{\pi r^{2}}, & x^{2}+y^{2} \leq r^{2},\\ 
0, & x^{2}+y^{2}>r^{2}.
\end{cases}
\]
试求概率 $P(X)\leq r/2$.
\end{example}
\begin{solution}
$p(x,y)$ 的非零区域与事件 $\{|X|\leq r/2\}$ 的交集部分见图~\ref{fig:3.1.4}, 因此所求概率为
\begin{align*}
	P(|X| \leq r / 2)&=\int_{-r / 2}^{r / 2} \int_{-\sqrt{r^{2}-x^{2}}}^{\sqrt{r^{2}-x^{2}}} \frac{1}{\pi r^{2}} \dd y \dd x
	=\frac{1}{\pi r^{2}} \int_{-r / 2}^{r / 2} 2 \sqrt{r^{2}-x^{2}} \dd x \\
	&=\frac{1}{\pi r^{2}}\left.\left[x \sqrt{r^{2}-x^{2}}+r^{2} \arcsin \frac{x}{r}\right]\right|_{-r / 2} ^{r / 2}	\\
	&=\frac{1}{\pi r^{2}}\left[r \sqrt{r^{2}-\frac{r^{2}}{4}}+2 r^{2} \arcsin \frac{1}{2}\right]	\\
	&=\frac{1}{\pi}\left[\frac{\sqrt{3}}{2}+\frac{\pi}{3}\right]=0.609	\\
\end{align*}
\end{solution}
\begin{figure}[htbp]
\centering
\includegraphics[width=0.4\textwidth]{fig3-1-4.png}
\caption{$p(x,y)$ 的非零区域与有关事件的交集部分}\label{fig:3.1.4}
\end{figure}

\textbf{四、二元正态分布}

如果二维随机变量 $(X,Y)$ 的联合密度函数 (见图~\ref{fig:3.1.5}) 为
	\begin{equation}\label{eq:3.1.8}
	\begin{aligned} 
	p(x, y) &=\frac{1}{2 \pi \sigma_{1} \sigma_{2} \sqrt{1-\rho^{2}}} \exp \bigg\{-\frac{1}{2\left(1-\rho^{2}\right)}
	\bigg[\frac{\left(x-\mu_{1}\right)^{2}}{\sigma_{1}^{2}}		\\
	&-2 \rho \frac{\left(x-\mu_{1}\right)\left(y-\mu_{2}\right)}{\sigma_{1} \sigma_{2}}+\frac{\left(y-\mu_{2}\right)^{2}}{\sigma_{2}^{2}} 
	\bigg] \bigg\}, \quad-\infty<x, y<+\infty \\
	\end{aligned}
	\end{equation}
则称 $(X,Y)$ 服从二元正态分布, 记为 $(X,Y)\sim N(\mu_1,\mu_2,\sigma_1^2,\sigma_2^2,\rho)$. 其中五个参数的取值范围分别是:
	\[
	 	-\infty<\mu_{1}, \mu_{2}<+\infty; \quad \sigma_{1}, \sigma_{2}>0 ; \quad-1 \leq \rho \leq 1.
	\]
	以后将指出: $\mu_1,\mu_2$ 分别是 $X$ 与 $Y$ 的均值, $\sigma_1^2,\sigma_2^2$ 分别是 $X$ 与 $Y$ 的方差, $\rho$ 是
	 $X$ 与 $Y$ 的相关系数.

	 二元正态密度函数的图形很像一顶四周无限延伸的草帽, 其中心点在 $(\mu_1,\mu_2)$ 处, 其等高线是椭圆.

	 \begin{figure}[htbp]
	 	\centering
	 	\includegraphics[width=0.4\textwidth]{fig3-1-5.png}
	 	\caption{二元正态密度函数}\label{fig:3.1.5}
	 \end{figure}

	 \begin{example}\label{exam:3.1.7}
	 	设二维随机变量 $(X,Y)\sim N(\mu_1,\mu_2,\sigma_1^2,\sigma_2^2,\rho)$, 求 $(x,Y)$ 落在区域
	 	\[
	 	 	D=\left\{(x, y) ; \frac{\left(x-\mu_{1}\right)^{2}}{\sigma_{1}^{2}}-2 \rho \frac{\left(x-\mu_{1}\right)\left(y-\mu_{2}\right)}{\sigma_{1} \sigma_{2}}+\frac{\left(y-\mu_{2}\right)^{2}}{\sigma_{2}^{2}} \leqslant \lambda^{2}\right\}
	 	\]
	 	内的概率.
	 \end{example}
	 \begin{solution}
	 	所求的概率为
	 	\begin{equation*}
			\begin{aligned} 
			p(x, y) &=\frac{1}{2 \pi \sigma_{1} \sigma_{2} \sqrt{1-\rho^{2}}} \iint_D \exp \bigg\{-\frac{1}{2\left(1-\rho^{2}\right)}
			\bigg[\frac{\left(x-\mu_{1}\right)^{2}}{\sigma_{1}^{2}}		\\
			&-2 \rho \frac{\left(x-\mu_{1}\right)\left(y-\mu_{2}\right)}{\sigma_{1} \sigma_{2}}+\frac{\left(y-\mu_{2}\right)^{2}}{\sigma_{2}^{2}} 
			\bigg] \bigg\} \dd x \dd y\\
			\end{aligned}
		\end{equation*}
		作变换
		\[
		 	\begin{cases}
		 		u =\frac{x-\mu_1}{\sigma_1}-\rho \frac{y-\mu_2}{\sigma_2},\\
		 		v =\frac{y-\mu_2}{\sigma_2}\sqrt{1-\rho^2}.
		 	\end{cases}
		\]
		则可得
		\[
		 	\frac{\partial(u, v)}{\partial(x, y)}= \begin{vmatrix}{\frac{1}{\sigma_{1}}} & {0} \\ {-\frac{\rho}{\sigma_{2}}} & {\frac{\sqrt{1-\rho^{2}}}{\sigma_{2}}}\end{vmatrix}=\frac{\sqrt{1-\rho^{2}}}{\sigma_{1} \sigma_{2}}, \quad|J|=\frac{\sigma_{1} \sigma_{2}}{\sqrt{1-\rho^{2}}},
		\]
		由此得
		\[
		 	p=\frac{1}{2 \pi(1-\rho^{2})} \iint_{x^{2}+v^{2} \leq \lambda^{2}} \exp \left\{-\frac{u^{2}+v^{2}}{2\left(1-\rho^{2}\right)}\right\} \dd u \dd v
		\]
		再作极坐标变换
		\[
		 	\begin{cases} 
		 		u=r \sin \alpha ,\\ 
		 		v=r \cos \alpha	,
		 	\end{cases}
		\]
		则可得
		\[
		 	J=\frac{\partial(u, v)}{\partial(r, \alpha)}= \begin{vmatrix}{\sin \alpha} & {\cos \alpha} \\ {r \cos \alpha} & {-r \sin \alpha}\end{vmatrix}=-r\left(\sin ^{2} \alpha+\cos ^{2} \alpha\right)=-r,
		\]
		最后得
		\[
		 	\begin{aligned} 
		 	p &=\frac{1}{2 \pi (1-\rho^{2})} \int_{0}^{2 \pi} \dd \alpha \int_{0}^{\lambda} r \exp \left\{-\frac{r^{2}}{2\left(1-\rho^{2}\right)}\right\} \dd r \\ 
		 	&=\int_{0}^{\lambda} \exp \left\{-\frac{r^{2}}{2\left(1-\rho^{2}\right)}\right\}\dd \left(\frac{r^{2}}{2\left(1-\rho^{2}\right)}\right)\\ 
		 	&=\left.-\exp \left\{-\frac{r^{2}}{2\left(1-\rho^{2}\right)}\right\}\right|_{0} ^{\lambda}=1-\exp \left\{-\frac{\lambda^{2}}{2\left(1-\rho^{2}\right)}\right\} .
		 	\end{aligned}
		\]
	 \end{solution}

	 \begin{xiti}
	 	\item 一批产品中有一等品 50\%, 二等品 30\%, 三等品 20\%. 从中有放回地抽取 5 件, 以 $X$、$Y$ 分别表示取出的 5 件中一等品、二等品的件数, 
	 	求 $(X,Y)$ 的联合分布列.
	 	\item 100 件产品中有 50 件一等品, 30 件二等品, 20 件三等品. 从中不放回地抽取 5 件, 以 $X$、$Y$ 分别表示取出的 5 件中一等品、二等品的件数,
	 	求 $(X,Y)$ 的联合分布列.
	 	\item 盘子里装有 3 只黑球、2 只红球、2 只白球, 从中任取 4 只, 以 $X$ 表示取到黑球的只数, 以 $Y$ 表示取到红球的只数,试求 $P\{X=Y\}$.
	 	\item 设随机变量 $X_i$,$i=1,2$, 的分布列如下, 且满足 $P(X_1X_2=0) =1$, 试求 $P(X_1=X_2)$.
	 	\begin{center}
	 		\begin{tabularx}{0.8\textwidth}{Z|ZZZ}
	 		$X_t$&	-1&		0&		1	\\
	 		\hline
	 		$P$	&	0.25&	0.5&	0.25	\\
	 	\end{tabularx}
	 	\end{center}
	 	\item 设随机变量 $(X,Y)$ 的联合密度函数为
	 	\[
	 	 	p(x, y)=\begin{cases}k(6-x-y),\quad & 0<x<2,2<y<4 \\
	 	 	 0,				&	\text{其他} .
	 	 	\end{cases}
	 	\]
	 	试求
	 	\begin{enumerate}[(1)]
	 		\item 常数 $k$;
	 		\item $P\{X<1,Y<3\}$;
	 		\item $P\{X<1.5\}$;
	 		\item $P\{X+Y\leq 4\}$.
	 	\end{enumerate}
	 	\item 设随机变量 $(X,Y)$ 的联合密度函数为
	 	\[
	 	 	p(x, y)=\begin{cases}
	 	 				k \ee^{-(3 x+4 y)},&	x>0,y>0 \\
	 	 	 			0,		&		\text{其他} . 
	 	 	 		\end{cases}
	 	\]
	 	试求
	 	\begin{enumerate}[(1)]
	 		\item 常数 $k$;
	 		\item $(X,Y)$ 的联合分布函数 $F(X,Y)$;
	 		\item $P\{0<X\leq 1,0<Y\leq 2\}$.
	 	\end{enumerate}
	 	\item 设二维随机变量 $(X,Y)$ 的联合密度函数为
	 	\[
	 	 	p(x, y)=\begin{cases}
	 	 				4 x y,&	0<x<1,0<y<1, \\ 
	 	 				0,&		\text{其他} .
	 	 			\end{cases}
	 	\]
	 	试求
	 	\begin{enumerate}[(1)]
	 		\item $P(0<X<0.5,0.25<Y<1)$;
	 		\item $P(X+Y)$;
	 		\item $P(X<Y)$;
	 		\item $(X,Y)$ 的联合分布函数.
	 	\end{enumerate}
	 	\item 设二维随机变量 $(X,Y)$ 的联合密度函数为
	 	\[
	 	 	p(x,y)=\begin{cases}
	 	 		k,&		0<x^2<y<x<1;\\
	 	 		0,&		\text{其他} .
	 	 	\end{cases}
	 	\]
	 	\begin{enumerate}[(1)]
	 		\item 试求常数 $k$;
	 		\item 求 $P(X>0.5)$ 和 $P(Y<0.5)$. 
	 	\end{enumerate}
	 	\item 设二维随机变量 $(X,Y)$ 的联合密度函数为
	 	\[
	 	 	p(x,y)=\begin{cases}
	 	 		6 (1-y),&	0<x<y<1;	\\
	 	 		0,&			\text{其他} .
	 	 	\end{cases}
	 	\]
	 	\begin{enumerate}[(1)]
	 		\item 求 $P(X>0.5,Y>0.5)$;
	 		\item 求 $P(X<0.5)$ 和 $P(Y<0.5)$;
	 		\item 求 $P(X+Y)<1$.
	 	\end{enumerate}
	 	\item 设随机变量 $Y$ 服从参数为 $\lambda=1$ 的指数分布, 定义随机变量 $X_k$ 如下
	 	\[
	 	 	X_k=\begin{cases}
	 	 		0,&		Y\leq k, \\
	 	 		1,&		Y>k,
	 	 	\end{cases} \quad k=1,2.
	 	\]
	 	求 $X_1$ 和 $X_2$ 的联合分布列.
	 	\item 设二维随机变量 $(X,Y)$ 的联合密度函数为
	 	\[
	 	 	p(x,y)=\begin{cases}
	 	 		x^2+\frac{xy}{3},&	0<x<1,0<y<2;\\
	 	 		0,		&			\text{其他} .
	 	 	\end{cases}
	 	\]
	 	求 $P(X+Y)\geq 1$.
	 	\item 设二维随机变量 $(X,Y)$ 的联合密度函数为
	 	\[
	 	 	p(x,y)=\begin{cases}
	 	 		\ee^{-y},	&	0<x<y;\\
	 	 		0&			\text{其他} .
	 	 	\end{cases}
	 	\]
	 	试求 $P(X,Y)\leq 1$.
	 	\item 设二维随机变量 $(X,Y)$ 的联合密度函数为
	 	\[
	 	 	p(x,y)=\begin{cases}
	 	 		1/2,&	0<x<1,0<y<2;\\
	 	 		0,&		\text{其他} .	
	 	 	\end{cases}
	 	\]
	 	求 $X$ 与 $Y$ 中至少一个小于 0.5 的概率.
	 	\item 从 (0,1) 中随机地取两个数, 求其积不小于 3/16, 且其和不大于 1 的概率.
	 \end{xiti}

	\section{边际分布与随机变量的独立性}\label{sec:3.2}
  二维联合分布函数(二维联合分布列、二维联合密度函数也一样)含有丰富的信息,\ 主要有以下三方面信息:
  \begin{itemize}
  	\item 每个分量的分布(每个分量的所有信息),\ 即边际分布.
  	\item 两个分量之间的关联程度,\ 即协方差和相关系数.
  	\item 给定一个分量时,\ 另一个分量的分布,\ 即条件分布.
  \end{itemize}
  我们的目的时将这些信息从联合分布中挖掘出来,\ 本节先讨论边际分布.
  
  \subsection{边际分布函数}\label{ssec:3.2.1}
  如果在二维随机变量 $(X,Y)$ 的联合分布函数 $F(X,Y)$ 中令 $y\to+\infty$ ,\ 由于 $|Y<+\infty|$ 为必然事件,\ 故可得
  \begin{equation*}
  \lim_{y\to+\infty}F(x,y)=P(X\leqslant x,Y<+\infty)=P(X\leqslant x),
  \end{equation*}
  这是一个分布函数,\ 被称为 $X$ 的边际分布,\ 记为
  \begin{equation}
  F_{X}(x)=F(x,+\infty).\label{eq:3.2.1}
  \end{equation}
  类似地,\ 在 $F(x,y)$ 中令 $x\to+\infty$ ,\ 可得 $Y$ 的边际分布
  \begin{equation}
  F_{Y}(y)=F(+\infty,y).\label{eq:3.2.2}
  \end{equation}
  在三维随机变量 $(X,Y,Z)$ 的联合分布函数 $F(x,y,z)$ 中,\ 用类似的方法可得到更多的边际分布函数:
  \begin{align*}
  &F_{X}(x) = F(x,+\infty,+\infty);\\
  &F_{Y}(y) = F(+\infty,y,+\infty);\\
  &F_{Z}(z) = F(+\infty,+\infty,z);\\
  &F_{X,Y}(x,y) = F(x,y,+\infty);\\
  &F_{X,Z}(x,z) = F(x,+\infty,z);\\
  &F_{Y,Z}(y,z) = F(+\infty,y,z).
  \end{align*}
  在更高维的场合,\ 也可类似地从联合分布函数获得其低维的边际分布函数.
  \begin{example}\label{exam:3.2.1}
  	设二维随机变量 $(X,Y)$ 的联合分布函数为
  	\begin{equation*}
  	F(x,y)=
  	\begin{cases}
  	1-\ee^{-x}-\ee^{-y}+\ee^{-x-y-\lambda xy}, & x>0,y>0;\\
  	0, & \text{其他}.
  	\end{cases}
  	\end{equation*}
  	这个分布被称为二维指数分布,\ 其中参数 $\lambda>0$.
  	
  	由此联合分布函数 $F(x,y)$ ,\ 容易获得 $X$ 与 $Y$ 的边际分布函数为
  	\begin{align*}
  	F_{X}(x) &= F(x,+\infty)=
  	\begin{cases}
  	1-\ee^{-x}, & x>0;\\
  	0, & x\leqslant0.
  	\end{cases}\\
  	F_{Y}(y) &= F(+\infty,y)=
  	\begin{cases}
  	1-\ee^{-y}, & y>0;\\
  	0, & y\leqslant0.
  	\end{cases}
  	\end{align*}
  	它们都是一维指数分布,\ 且与参数 $\lambda>0$ 无关.\ 不同的 $\lambda>0$ 对应不同的二维指数分布,\ 但它们的两个边际分布不变.\ 这说明:\ 二维联合分布不仅含有每个分量的概率分布,\ 而且还含有两个变量 $X$ 与 $Y$ 间关系的信息,\ 这正是人们要研究多维随机变量的原因.
  \end{example}
  
  \subsection{边际分布列}\label{ssec:3.2.2}
  在二维离散随机变量 $(X,Y)$ 的联合分布列 $\left\{P(X=x_i,Y=y_i)\right\}$ 中,\ 对 $j$ 求和所得的分布列
  \begin{equation}
  \sum_{j=1}^{+\infty}P(X=x_i,Y=y_j)=P(X=x_i),\ i=1,2,\ldots\label{eq:3.2.3}
  \end{equation}
  被称为 $X$ 的边际分布列.\ 类似地,\ 对 $i$ 求和所得的分布列
  \begin{equation}
  \sum_{i=1}^{+\infty}P(X=x_i,Y=y_j)=P(Y=y_j),\ j=1,2,\ldots\label{eq:3.2.4}
  \end{equation}
  被称为 $Y$ 的边际分布列.
  \begin{example}\label{exam:3.2.2}
  	设二维随机变量 $(X,Y)$ 有如下的联合分布列
  	\begin{equation*}
  	\begin{tabularx}{0.8\textwidth}{ZZZZ}
  	\toprule
  	 & \multicolumn{3}{c}{$Y$}\\
  	\cmidrule{2-4}
  	$X$ & 1 & 2 & 3\\
  	\midrule
  	0 & 0.09 & 0.21 & 0.24\\
  	1 & 0.07 & 0.12 & 0.27\\
  	\bottomrule
  	\end{tabularx}
  	\end{equation*}
  	求 $X$ 与 $Y$ 的边际分布列.
  	\end{example}
  
  \begin{solution}
  	在上述联合分布列中,\ 对每一行求和得 $0.54$ 与 $0.46$ ,\ 并把它们写在对应行得右侧,\ 这就是 $X$ 的边际分布列.\ 再对每一列求和,\ 得 $0.16,0.33$ 和 $0.51$ ,\ 并把它们写在对应列的下侧,\ 这就是 $Y$ 得边际分布列.
  	\begin{equation*}
  	\begin{tabularx}{0.8\textwidth}{ZZZZZ}
  	\toprule
  	 & \multicolumn{3}{c}{$Y$} & \\
  	 \cmidrule{2-4}
  	$X$ & 1 & 2 & 3 & $P(X=i)$ \\
  	\midrule
  	0 & 0.09 & 0.21 & 0.24 & 0.54 \\
  	1 & 0.07 & 0.12 & 0.27 & 0.46 \\
  	\midrule
  	$P(Y=j)$ & 0.16 & 0.33 & 0.51 &1\\
  	\bottomrule
  	\end{tabularx}
  	\end{equation*}
  \end{solution}

  \subsection{边际密度函数}\label{ssec:3.2.3}
  如果二维连续随机变量 $(X,Y)$ 的联合密度函数为 $p(x,y)$ ,\ 因为
  \begin{align*}
  F_{X}(x) &= F(x,+\infty)=\int_{-\infty}^{x}\left( \int_{-\infty}^{+\infty}p(u,v)\,\dd v \right)\,\dd u=\int_{-\infty}^{x}p_{X}(u)\,\dd u,\\
  F_{Y}(y) &= F(+\infty,y)=\int_{-\infty}^{y}\left( \int_{-\infty}^{+\infty}p(u,v)\,\dd u \right)\,\dd v=\int_{-\infty}^{y}p_{Y}(v)\,\dd v,
  \end{align*}
  其中 $p_{X}(x)$ 和 $p_{Y}(y)$ 分别为
  \begin{align}
  p_{X}(x) &= \int_{-\infty}^{+\infty}p(x,y)\,\dd y.\label{eq:3.2.5}\\
  p_{Y}(y) &= \int_{-\infty}^{+\infty}p(x,y)\,\dd x.\label{eq:3.2.6}
  \end{align}
  它们恰好处于密度函数位置,\ 故称上式给出的 $p_{X}(x)$ 为 $X$ 的边际密度函数,\ $p_{Y}(y)$ 为 $Y$ 的边际密度函数.
  
  由联合密度函数求边际密度函数时,\ 要注意积分区域的确定.
  \begin{example}\label{exam:3.2.3}
  	设二维随机变量 $(X,Y)$ 的联合密度函数为
  	\begin{equation*}
  	p(x,y)=
  	\begin{cases}
  	1, & 0<x<1,|y|<x;\\
  	0, & \text{其他}.
  	\end{cases}
  	\end{equation*}
  	试求:(1)边际密度函数 $p_{X}(x)$ 和 $p_{Y}(y)$;(2) $P(X<1/2)$ 及 $P(Y>1/2)$ .
  \end{example}
  \begin{solution}
  	首先识别 $p(x,y)$ 的非零区域,\ 它如图~\ref{fig:3.2.1} 所示.
  	\begin{figure}[htbp]
  		\centering
  		\includegraphics[scale=0.5]{fig3-2-1.png}
  		\caption{$p(x,y)$ 的非零区域}\label{fig:3.2.1}
  	\end{figure}
  	\begin{itemize}
  		\item[(1)]求 $p_X(x)$:\ 当 $x\leqslant0$ 或 $x\geqslant1$ 时,\ 有 $p_X(x)=0$.\ 而当 $0<x<1$ 时,\ 有
  		\begin{equation*}
  		p_X(x)=\int_{-\infty}^{+\infty}p(x,y)\,\dd y=\int_{-x}^{x}\,\dd y=2x.
  		\end{equation*}
  		所以 $X$ 的边际密度函数为(见图~\ref{fig:3.2.2})
  		\begin{equation*}
  		p_X(x)=
  		\begin{cases}
  		2x, & 0<x<1;\\
  		0, & \text{其他}.
  		\end{cases}
  		\end{equation*}
  		\begin{figure}[h]
  			\centering
  			\includegraphics[scale=0.2]{fig3-2-2.png}
  			\caption{$X$ 的边际密度函数}\label{fig:3.2.2}
  		\end{figure}
  		再求 $p_Y(y):$\ 当 $y\leqslant-1$ 或 $y\geqslant1$ 时,\ 有 $p_Y(y)=0$.\ 而当 $-1<y<0$ 时,\ 有
  		\begin{equation*}
  		p_Y(y)=\int_{-\infty}^{+\infty}p(x,y)\,\dd x=\int_{-y}^{1}\,\dd x=1+y,
  		\end{equation*}
  		当 $0<y<1$ 时,\ 有
  		\begin{equation*}
  		p_Y(y)=\int_{-\infty}^{+\infty}p(x,y)\,\dd x=\int_{y}^{1}\,\dd x=1-y.
  		\end{equation*}
  		所以 $Y$ 的边际密度函数为(见图~\ref{fig:3.2.3})
  		\begin{equation*}
  		p_{Y}(y)=
  		\begin{cases}
  		1+y, & -1<y<0,\\
  		1-y, & 0<y<1,\\
  		0, & \text{其他}.
  		\end{cases}
  		\end{equation*}
  		\begin{figure}[h]
  			\centering
  			\includegraphics[scale=0.5]{fig3-2-3.png}
  			\caption{ $Y$ 的边际密度函数}\label{fig:3.2.3}
  		\end{figure}
  		\item[(2)] 要求的概率分别为
  		\begin{align*}
  		P(X<1/2) &= \int_{-\infty}^{1/2}p_{X}(x)\,\dd x=\int_{0}^{1/2}2x\,\dd x=\frac{1}{4}.\\
  		P(Y>1/2) &= \int_{1/2}^{+\infty}p_{Y}(y)\,\dd y=\int_{1/2}^{1}(1-y)\,\dd y=\frac{1}{8}.
  		\end{align*}
  	\end{itemize}
  \end{solution}
  
  \begin{example}\label{exam:3.2.4}
  	多项分布的边际分布仍为多项分布
  \end{example}
  \begin{solution}
  	下面只证三项分布的边际分布为二项分布.\ 设 $(X,Y)$ 服从三项分布 $M(n,p_1,p_2,p_3),$ \ 其联合分布列为
  	\begin{equation*}
  	P(X=i,Y=j)=\frac{n!}{i!j!(n-i-j)!}p_1^ip_2^j(1-p_1-p_2)^{n-i-j},\ i,j=1,2,\ldots,n,i+j\leqslant n.
  	\end{equation*}
  	对上式分别乘以和除以 $(1-p_1)^{n-i}/(n-i)!$ ,\ 再对 $j$ 从 $0$ 到 $n-1$ 求和,\ 并记 $p_2'=p_2/(1-p_1)$,\ 则可得
  	\begin{align*}
  	&\sum_{j=0}^{n-i}P(X=i,Y=j) = \frac{n!}{i!(n-i)!}p_1^i(1-p_1)^{n-i}.\\
  	&\sum_{j=0}^{n-i} \binom{n-i}{j} \left(\frac{p_2}{1-p_1}\right)^{j}\left(1-\frac{p_2}{1-p_1}\right)^{n-i-j} \\
  	&= \frac{n!}{i!(n-i)!}p_1^i(1-p_1)^{n-i}\left[p_2'+(1-p_2')\right]^{n-i}\\
  	&= \frac{n!}{i!(n-i)!}p_1^i(1-p_1)^{n-i}.
  	\end{align*}
  	所以 $X\sim b(n,p_1)$ .\ 同理可证 $Y\sim b(n,p_2)$.
  	
  	用类似的方法可以证明:\ 若 $(X_,,X_2,\ldots,X_r)\sim M(n,p_1,p_2,\ldots,p_r)$,\ 则 $X_i\sim b(n,p_i),\ i=1,2,\ldots,r$.
  \end{solution}
  
  \begin{example}\label{exam:3.2.5}
  	二维正态分布的边际分布为一维正态分布
  \end{example}
  \begin{solution}
  	设 $(X,Y)\sim N(\mu_1,\mu_2,\sigma_1^2,\sigma_2^2,\rho)$ .\ 先把 ~\ref{eq:3.1.8} 式二维正态密度函数 $p(x,y)$ 的指数部分
  	\begin{equation*}
  	-\frac{1}{2(1-\rho^2)}\left[\frac{(x-\mu_1)^2}{\sigma_1^2}-2\rho\frac{(x-\mu_1)(y-\mu_2)}{\sigma_1\sigma_2}+\frac{(y-\mu_2)^2}{\sigma_2^2}\right]
  	\end{equation*}
  	改写成
  	\begin{equation*}
  	-\frac{1}{2}\left[\rho\frac{x-\mu_1}{\sigma_1\sqrt{1-\rho^2}}-\frac{y-\mu_2}{\sigma_2\sqrt{1-\rho^2}}\right]^2-\frac{(x-\mu_1)^2}{2\sigma_1^2}.
  	\end{equation*}
  	再对积分
  	\begin{equation*}
  	\int_{-\infty}^{+\infty}\exp\left\{-\frac{1}{2}\left[\rho\frac{x-\mu_1}{\sigma_1\sqrt{1-\rho^2}}-\frac{y-\mu_2}{\sigma_2\sqrt{1-\rho^2}}\right]^2\right\}\,\dd y
  	\end{equation*}
  	作变换(注意把 $x$ 看作常量)
  	\begin{equation*}
  	t=\rho\frac{x-\mu_1}{\sigma_1\sqrt{1-\rho^2}}-\frac{y-\mu_2}{\sigma_2\sqrt{1-\rho^2}},
  	\end{equation*}
  	则
  	\begin{align*}
  	p_{X}(x) &= \int_{-\infty}^{+\infty}p(x,y)\,\dd y\\
  	&=\frac{1}{2\uppi\sigma_1\sigma_2\sqrt{1-\rho^2}}\exp\left\{-\frac{(x-\mu_1)^2}{2\sigma_1^2}\right\}\sigma_2\sqrt{1-\rho^2}\int_{-\infty}^{+\infty}\exp\left\{-\frac{t^2}{2}\right\}\,\dd t.
  	\end{align*}
  	注意到上式中的积分恰好等于 $\sqrt{2\uppi}$ ,\ 所以有
  	\begin{equation*}
  	p_{X}(x)=\frac{1}{\sqrt{2\uppi}\sigma_1}\exp\left\{-\frac{(x-\mu_1)^2}{2\sigma_1^2}\right\}.
  	\end{equation*}
  	这正是一维正态分布 $N(\mu_1,\sigma_1^2)$ 的密度函数,\ 即 $X\sim N(\mu_1,\sigma_1^2)$ .\ 同理可证 $Y\sim N(\mu_2,\sigma_2^2)$ .\ 由此可见
  	\begin{itemize}
  		\item 二维正态分布的边际分布中不含参数 $\rho$ .
  		\item 这说明二维正态分布 $N(\mu_1,\mu_2,\sigma_1^2,\sigma_2^2,0.1)$ 与 $N(\mu_1,\mu_2,\sigma_1^2,\sigma_2^2,0.2)$ 的边际分布是相同的.
  		\item 具有相同边际分布的多维联合分布可以是不同的.
  	\end{itemize}
  \end{solution}
  
   \subsection{随机变量间的独立性}\label{ssec:3.2.4}
   在多维随机变量中,\ 各分量的取值有时会相互影响,\ 但有时毫无影响.\ 譬如一个人的身高 $X$ 和体重 $Y$ 就会相互影响,\ 但与收入 $Z$ 一般无影响.\ 当两个随机变量取值的规律互不影响时,\ 就称它们是相互独立的.
   \begin{definition}{}{3.2.1}
   	设 $n$ 维随机变量 $(X_1,X_2,\ldots,X_n)$ 的联合分布函数为 $F(x_1,x_2,\ldots,x_n)$ ,\ $F_i(x_i)$ 为 $X_i$ 的边际分布函数.\ 如果对任意 $n$ 个实数 $x_1,x_2,\ldots,x_n$ ,\ 有
   	\begin{equation}\label{eq:3.2.7}
   		F(x_1,x_2,\ldots,x_n)=\prod_{i=1}^{n}F_i(x_i),
   	\end{equation}
   	则称 $X_1,X_2,\ldots,X_n$ 相互独立.
   \end{definition}
   在离散随机变量场合,\ 如果对其任意 $n$ 个取值 $x_1,x_2,\ldots,x_n$ ,\ 有
   \begin{equation}\label{eq:3.2.8}
   	P(X_1=x_1,X_2=x_2,\ldots,X_n=x_n)=\prod_{i=1}^{n}P(X_i=x_i),
   \end{equation}
   则称 $X_1,X_2,\ldots,X_n$ 相互独立.
   
   在连续随机变量场合,\ 如果对任意 $n$ 个实数 $x_1,x_2,\ldots,x_n$ ,\ 有
   \begin{equation}\label{eq:3.2.9}
   	p(x_1,x_2,\ldots,x_n)=\prod_{i=1}^{n}p_i(x_i),
   \end{equation}
   则称 $X_1,X_2,\ldots,X_n$ 相互独立.
   
   \begin{example}
   	设 $(X,Y)$ 是二维离散随机变量,\ $X$ 和 $Y$ 的边际分布列分别如下所示:
   	\begin{table}[h]
   		\centering
   		\begin{tabular}[h]{c|ccc}
   			\hline
   			$X$ & $-1$ & $0$ & $1$\\
   			\hline
   			$P$ & $1/4$ & $1/2$ & $1/4$\\
   			\hline
   		\end{tabular}
   	    \qquad
   	    \begin{tabular}[h]{c|cc}
   	    	\hline
   	    	$Y$ & $0$ & $1$\\
   	    	\hline
   	    	$P$ & $1/2$ & $1/2$\\
   	    	\hline
   	    \end{tabular}
   	\end{table}
   如果 $P\left\{XY=0\right\}=1$,\ 试求
   
   (1) $(X,Y)$ 的联合分布列;
   
   (2) $X$ 与 $Y$ 是否独立?

   \end{example}
   \begin{solution}
   	\begin{itemize}
   		\item[(1)] 记 $(X,Y)$ 得联合分布列如下,\ 其中 $p_{ij}=P(X=i,Y=j)$ ,\ 在联合分布列的右侧是 $X$ 的边际分布列,\ 下侧是 $Y$ 的边际分布列.
   		\begin{center}
   			\begin{tabularx}{0.8\textwidth}{XXXX}
   				\toprule
   				 & \multicolumn{2}{c}{$Y$} & \\
   				\cmidrule{2-3}
   				$X$ & $0$ & $1$ & $P(X=i)$\\
   				\midrule
   				$-1$ & $p_{11}$ & $p_{12}$ & $1/4$\\
   				$0$ & $p_{21}$ & $p_{22}$ & $1/2$\\
   				$1$ & $p_{31}$ & $p_{32}$ & $1/4$\\
   				\midrule
   				$P(Y=j)$ & $1/2$ & $1/2$ & $1$\\
   				\bottomrule
   			\end{tabularx}
   		\end{center}
   		由 $P(XY=0)=1$ ,\ 知 $P(XY\ne0)=0$ ,\ 即
   		\begin{equation*}
   		p_{12}=P(X=-1,Y=1)=0,\quad p_{32}=P(X=1,Y=1)=0.
   		\end{equation*}
   		其余四个概率可由下面等式分别确定.
   		\begin{center}
   			\begin{tabular}{l}
   				从表中第一行看,\ 由 $p_{11}+p_{12}=1/4$ ,\ 得 $p_{11}=1/4$ .\\
   				从表中第三行看,\ 由 $p_{31}+p_{32}=1/4$ ,\ 得 $p_{31}=1/4$ .\\
   				从表中第一列看,\ 由 $p_{11}+p_{21}+p_{31}=1/2=1/4+p_{21}+1/4$ ,\ 得 $p_{21}=0$ .\\
   				从表中第二列看,\ 由 $p_{12}+p_{22}+p_{32}=1/2=0+p_{22}+0$ ,\ 得 $p_{22}=1/2$ .
   			\end{tabular}
   		\end{center}
   		
   		于是得 $(X,Y)$ 的联合分布列如下:
   		\begin{center}
   			\begin{tabularx}{0.8\textwidth}{XXXX}
   				\toprule
   				 & \multicolumn{2}{c}{$Y$} & \\
   				\cmidrule{2-3}
   				$X$ & 0 & 1 & $p_{i\cdot}$\\
   				\midrule
   				$-1$ & $1/4$ & $0$ & $1/4$\\
   				$0$ & $0$ & $1/2$ & $1/2$\\
   				$1$ & $1/4$ & $0$ & $1/4$\\
   				\midrule
   				$p_{\cdot j}$ & $1/2$ & $1/2$ & $1$\\
   				\bottomrule
   			\end{tabularx}
   		\end{center}
   		\item[(2)] 因为 $P(X=0,Y=0)=p_{21}=0$ ,\ 而 $P(X=0)P(Y=0)=1/4$ ,\ 所以 $X$ 与 $Y$ 不独立.
   	\end{itemize}
   \end{solution}

   \begin{example}\label{exam:3.2.7}
   	若 $(X,Y)$ 的联合密度函数为
   	\begin{equation*}
   		p(x,y)=
   		\begin{cases}
   		8xy, & 0\leqslant x\leqslant y\leqslant1;\\
   		0, & \text{其他}.
   		\end{cases}
   	\end{equation*}
   	问 $X$ 与 $Y$ 是否相互独立?
   \end{example}
   \begin{solution}
   	为判断 $X$ 与 $Y$ 是否独立,\ 只需看边际密度函数的乘积是否等于联合密度函数.\ 为此先求边际密度函数.
   	当 $x<0$ 或 $x>1$ 时,\ $p_{X}(x)=0$ .\ 而当 $0\leqslant x\leqslant1$ 时,\ 有
   	\begin{equation*}
   		p_{X}(x)=\int_{x}^{1}8xy\,\dd y=8x\left(\frac{1}{2}-\frac{x^2}{2}\right)=4x\left(1-x^2\right).
   	\end{equation*}
   	因此
   	\begin{equation*}
   		p_{X}(x)=
   		\begin{cases}
   		4x\left(1-x^2\right), & 0\leqslant x\leqslant1,\\
   		0, & \text{其他}.
   		\end{cases}
   	\end{equation*}
   	同样,\ 当 $y<0$ 或 $y>1$ 时,\ $p_{Y}(y)=0$ .\ 而当 $0\leqslant y\leqslant 1$ 时,\ 有
   	\begin{equation*}
   		p_{Y}(y)=\int_{0}^{x}8xy\,\dd x=4y^3.
   	\end{equation*}
   	因此
   	\begin{equation*}
   		p_{Y}(y)=
   		\begin{cases}
   		4y^3, & 0\leqslant y\leqslant1;\\
   		0, & \text{其他}.
   		\end{cases}
   	\end{equation*}
   	由此得 $p(x,y)\ne p_{X}(x)p_{Y}(y)$ ,\ 所以 $X$ 与 $Y$ 不独立.
   \end{solution}

   \begin{example}\label{exam:3.2.8}
   	从 $(0,1)$ 中任取两个数,\ 求下列事件的概率.
   	
   	(1)两数之和小于 $1.2$ ;
   	
   	(2)两数之积小于 $1/4$ .
   \end{example}
   \begin{solution}
   	分别记这两个数为 $X$ 和 $Y$ ,\ 则 $X$ 和 $Y$ 独立,\ 且都服从 $(0,1)$ 上的均匀分布,\ $(X,Y)$ 的联合密度函数为
   	\begin{equation*}
   		p(x,y)=p_{X}(x)p_{Y}(y)=
   		\begin{cases}
   		1, & 0<x<1,0<y<1;\\
   		0, & \text{其他}.
   		\end{cases}
   	\end{equation*}
   	\begin{itemize}
   		\item[(1)] 从图~\ref{fig:3.2.4a}可知
   		\begin{align*}
   			P(X+Y<1.2)=\int_{0}^{0.2}\int_{0}^{1}\,\dd y\dd x+\int_{0.2}^{1}\int_{0}^{1.2-x}\,\dd y\dd x\\
   			=0.2+\int_{0.2}^{1}(1.2-x)\,\dd x=0.2+0.48=0.68\ .
   		\end{align*}
   		\begin{figure}[h]
   			\centering
   			\subfloat[$\{x+y<1.2,0<x,y<1\}$]{\label{fig:3.2.4a}
   			\includegraphics[scale=0.5]{fig3-2-4a.png}}
   			\qquad
   			\subfloat[$\{xy<1/4,0<x,y<1\}$]{\label{fig:3.2.4b}
   			\includegraphics[scale=0.5]{fig3-2-4b.png}
   			}
   			\caption{ $p(x,y)$ 的非零区域与有关事件的交集部分}\label{fig:3.2.4}
   		\end{figure}
   	\item[(2)] 从图~\ref{fig:3.2.4b}可知
   	\begin{align*}
   		P(XY<1/4)=\int_{0}^{1/4}\int_{0}^{1}\,\dd y\dd x+\int_{1/4}^{1}\int_{0}^{1/(4x)}\,\dd y\dd x\\
   		=\frac{1}{4}+\int_{1/4}^{1}\frac{1}{4x}\,\dd x=\frac{1}{4}+\frac{1}{4}\ln4=0.5966\ .
   	\end{align*}
   	\end{itemize}
   \end{solution}

   \begin{xiti}
   		\item 设二维离散随机变量 $(X,Y)$ 的可能值为
   		\begin{equation*}
   			(0,0),\ (-1,1),\ (-1,2),\ (1,0).
   		\end{equation*}
   		且取这些值的概率依次为 $1/6,1/3,1/12,5/12$ ,\ 试求 $X$ 与 $Y$ 各自的边际分布列.
   		\item 设二维随机变量 $(X,Y)$ 的联合分布函数为
   		\begin{equation*}
   			F(x,y)=\begin{cases}
   			1-\ee^{-\lambda_{1}x}-\ee^{-\lambda_{2}y}+\ee^{-\lambda_{1}x-\lambda_{2}y-\lambda_{12}\max\{x,y\}}, & x>0,y>0;\\
   			0, & \text{其他}.
   			\end{cases}
   		\end{equation*}
   		试求 $X$ 和 $Y$ 各自的边际分布函数.
   		\item 试求以下二维均匀分布的边际分布:
   		\begin{equation*}
   			p(x,y)=\begin{cases}
   			\frac{1}{\uppi}, & x^2+y^2\leqslant1;\\
   			0, & \text{其他}.
   			\end{cases}
   		\end{equation*}
   		\item 设平面区域 $D$ 由曲线 $y=1/x$ 及直线 $y=0,x=1,x=\ee^2$ 所围成,\ 二维随机变量 $(X,Y)$ 在区域 $D$ 上服从均匀分布,\ 试求 $X$ 的边际密度函数.
   		\item 求以下给出的 $(X,Y)$ 的联合密度函数的边际密度函数 $p_{X}(x)$ 和 $p_{Y}(y)$.
   		\begin{align*}
   			p_{1}(x,y) &= \begin{cases}
   			\ee^{-y}, & 0<x<y;\\
   			0, & \text{其他}.
   			\end{cases}\\
   			p_{2}(x,y) &= \begin{cases}
   			\frac{5}{4}(x^2+y), & 0<y<1-x^2;\\
   			0, & \text{其他}.
   			\end{cases}
   		\end{align*}
   		\item 设二维随机变量 $(X,Y)$ 的联合密度函数为
   		\begin{equation*}
   			p(x,y)=\begin{cases}
   			6, & 0<x^2<y<x<1;\\
   			0, & \text{其他}.
   			\end{cases}
   		\end{equation*}
   		试求边际密度函数 $p_{X}(x)$ 和 $p_{Y}(y)$ .
   		\item 试验证:\ 以下给出的两个不同的联合密度函数,\ 它们有相同的边际密度函数.
   		\begin{align*}
   			p(x,y) &= \begin{cases}
   			x+y, & 0\leqslant x\leqslant1,0\leqslant y\leqslant1;\\
   			0, & \text{其他}.
   			\end{cases}\\
   			g(x,y) &= \begin{cases}
   			(0.5+x)(0.5+y), & 0\leqslant x\leqslant1,0\leqslant y\leqslant1;\\
   			0, & \text{其他}.
   			\end{cases}
   		\end{align*}
   		\item 设随机变量 $X$ 和 $Y$ 独立同分布,\ 且
   		\begin{equation*}
   			P(X=-1)=P(Y=-1)=P(X=1)=P(Y=1)=\frac{1}{2}
   		\end{equation*}
   		试求 $P\{X=Y\}$ .
   		\item 甲、乙两人独立地各进行两次射击,\ 假设甲的命中率为 $0.2$ ,\ 乙的命中率为 $0.5$ ,\ 以 $X$ 和 $Y$ 分别表示甲和乙的命中次数,\ 试求 $P(X\leqslant Y)$ .
   		\item 设随机变量 $X$ 和 $Y$ 相互独立,\ 其联合分布列为
   		\begin{center}
   			\begin{tabularx}{0.8\textwidth}{Z|ZZZ}
   				\hline
   				 & \multicolumn{3}{c}{ $Y$ }\\
   				\cline{2-4}
   				$X$ & $y_1$ & $y_2$ & $y_3$\\
   				\hline
   				$x_1$ & $a$ & $1/9$ & $c$\\
   				\hline
   				$x_2$ & $1/9$ & $b$ & $1/3$\\
   				\hline
   			\end{tabularx}
   		\end{center}
   	
   	    试求联合分布列中的 $a,b,c$ .
   		\item 设 $k_1,k_2$ 分别是掷一枚骰子两次先后出现的点数.\ 试求方程 $x^2+k_1x+k_2=0$ 有实根的概率 $p$ 和有重根的概率 $q$.
   		\item 设 $X$ 与 $Y$ 是两个相互独立的随机变量,\ $X\sim U(0,1)$ ,\ $Y\sim Exp(1)$ .\ 试求
   		\begin{itemize}
   			\item[(1)] $X$ 与 $Y$ 的联合密度函数;
   			\item[(2)] $P(Y\leqslant X)$ ;
   			\item[(3)] $P(X+Y\leqslant1)$ .
   		\end{itemize}
   	    \item 设随机变量 $(X,Y)$ 的联合密度函数为
   	    \begin{equation*}
   	    	p(x,y)=\begin{cases}
   	    	3x, & 0<x<1,0<y<x;\\
   	    	0, & \text{其他}.
   	    	\end{cases}
   	    \end{equation*}
   	    试求
   	    \begin{itemize}
   	    	\item[(1)] 边际密度函数 $p_{X}(x)$ 和 $p_{Y}(y)$ ;
   	    	\item[(2)] $X$ 与 $Y$ 是否独立?
   	    \end{itemize}
       \item 设随机变量 $(X,Y)$ 的联合密度函数为
       \begin{equation*}
       	p(x,y)=\begin{cases}
       	1, & |x|<y,0<y<1;\\
       	0, & \text{其他}.
       	\end{cases}
       \end{equation*}
       试求
       \begin{itemize}
       	\item[(1)] 边际密度函数 $p_{X}(x)$ 和 $p_{Y}(y)$ ;
       	\item[(2)] $X$ 与 $Y$ 是否独立?
       \end{itemize}
       \item 在长为 $a$ 的线段的中点的两边随机地各选取一点,\ 求两点间的距离小于 $a/3$ 的概率.
       \item 设二维随机变量 $(X,Y)$ 的联合密度函数为 $p(x,y)$ .\ 证明:\ $X$ 与 $Y$ 相互独立的充要条件是 $p(x,y)$ 可分离变量,\ 即 $p(x,y)=h(x)g(y)$ .\ 又问 $h(x),g(y)$ 与边际密度函数有什么关系?
   \end{xiti}