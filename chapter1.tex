% Edit by:曹甄强 
\chapter{随机事件与概率}
\section{随机事件及其运算}
\subsection{随机现象}

概率论与数理统计研究的对象是随机对象

在一定的条件下, 并不总是出现相同结果的现象称为随机现象, 如抛一枚硬币与掷一颗骰子.随机现象有两个特点: 

\begin{enumerate}
	\item 结果不止一个.
	\item 哪一个结果出现, 人们事先并不知道
\end{enumerate}

只有一个结果的现象称为\textbf{确定性现象}.例如, 每天早晨太阳从东方升起;水在标准大气压(~压力约为101kPa~)下加热到100$\circ$C就沸腾;一个口袋中有十只完全相同的白球, 从中任取一支必然为白球.

\begin{example}
	
	随机现象的例子
	
	\begin{enumerate}
		\item 抛一枚硬币, 有可能正面朝上, 也有可能反面朝上;
		\item 掷一颗骰子, 出现的点数;
		\item 一天内进入某超市的顾客数;
		\item 某种型号电视机的寿命;
		\item 测量某物理量(~长度、直径等~)的误差.
	\end{enumerate}
	\label{exam:1.1.1}
\end{example}

随机现象到处可见
	
在相同条件下可以重复的随机现象又称为\textbf{随机试验}.也有很多随机现象是不能重复的, 例如某场足球赛的输赢是不能重复的, 某些经济现象(~如失业、经济增长速度等~)也不能重复, 概率论与数理统计主要研究能大量重复的随机现象, 但也十分注意研究不能重复的随机现象.
	
\subsection{样本空间}
	
随机现象的一切可能基本结果组成的集合称为样本空间, 记为$\Omega=|\omega|$,其中$\omega$表示基本结果, 又称为\textbf{样本点}.样本点是今后抽样的\textbf{最基本单元}.认识随机现象首先要列出它的样本空间.
	
\begin{example}
	
	下面给出~\ref{exam:1.1.1}中随机现象的样本空间.
	\begin{enumerate}
		\item 抛一枚硬币的样本空间为: $\Omega_1=\{\omega_{1}+\omega_{2}|$, 其中$\omega_{1}\}$表示正面朝上, $\omega_{2}$表示反面朝上
		\item 掷一颗骰子的样本空间为$ \Omega_{2}=\{\omega_{1},\omega_{2},\cdots,\omega_{6}\} $, 其中$ \omega_{i} $表示出现$ i $点, $ i=1,2,\cdots,6 $. 也更直接明了地记此样本空间为: $ \Omega_{2}=\{1,2,\cdots,6\} $.
		\item 一天内进入某商场地顾客数地样本空间为: 
		\[\Omega_{3}=\{0,1,2,\cdots,500,\cdots,10^4,\cdots\},\]
		其中“0”表示“一天内无人光顾此商场”, 面“$10^4$”表示“一天内有一万人光顾此商场”.虽然此两种情况很少发生, 但我们无法说此两种情况不可能发生, 甚至于我们不能确切地说出一天内进入该商场地最多人数, 所以此样本空间用非负整数集表示,既不脱离实际情况, 又是合理抽象, 便于数学上地处理.
		\item 电视机寿命地样本空间为: $ \Omega_{4}=\{t,t \leq 0\} $.
		\item 测量误差地样本空间为:$ \Omega_{5}=\{x,-\inf < x < +\inf\} $.
	\end{enumerate}
\end{example}   

需要注意的是:
\begin{enumerate}
	\item 样本空间中的元素可以是数也可以不是数.
	\item 样本空间至少有两个样本点,仅含两个样本点的样本空间是最简单的相本空间.
	\item 从样本空间含有样本点的个数来区分,样本空间可分为有限与无限两类,譬如以上样本空间$\Omega_1$和$\Omega_{2}$中样本点的个数为有限个,而$\Omega_3$、$\Omega_4$,及$\Omega_5$中样本点的个数为无限个.但$\Omega_3$,中样本点的个敷为可列个,而$\Omega_4$和$\Omega_5$,中的元素个数为不可列无限个,在以后的数学处理上我们往往将样本点的个数为有限个或可列个的情况归为一类,称为高散样本空间,而将样本点的个数为不可列无限个的情况归为另一类,称为连续样本空间,由于这两类样本空间有着本质上的差异,故分别称呼之.
\end{enumerate}

\subsection{随机事件}
随机现象的某些样本点组成的集合称为随机事件,简称事件,常用大写字母$A,B,C,\cdots$ 表示,如在掷一颗骰子中,$A=$ “出现奇数点” 是一个事件,即$A=\{1,3,5\}$.

在以上事件的定义中,要注意以下几点.
\begin{enumerate}
	\item 任一事件A是相应样本空间的一个子集.在概率论中常用一个长方形表示样本空间 $\Omega$,用其中一个圆或其他几何图形表示事件A,见图1.1.1,这类图形称为\textbf{维恩(Venn)图}.
	\item 当子集A中某个样本点出现了,就说事件A发生了,或者说事件A发生当且仅当A中某个样本点出现了.
	\item 事件可以用集合表示,也可用明白无误的语言描述.
	\item 由样本空间 $\Omega$ 中的单个元素组成的子集称为\textbf{基本事件},而样本空间 $\Omega$ 的最大子集(即 $\Omega$ 本身)称为\textbf{必然事件},样本空间 $\Omega$ 的最小子集(即空集$\oslash$)称为不可能事件.
\end{enumerate}
%pic111 事件A的维恩图

\begin{example}
	
\end{example}




















\subsection{随机变量}
\subsection{事件间的关系}
\subsection{事件运算}
\subsection{事件域}

\section{概率的定义及其确定方法}
\subsection{概率的定义及其确定方法}
\subsection{概率的公式化定义}
\subsection{确定概率的频率方法}
\subsection{确定概率的古典方法}
\subsection{确定概率的几何方法}
\subsection{确定概率的主观方法}

\section{概率的性质}
\subsection{概率的可加性}
\subsection{概率的单调性}
\subsection{概率的加法公式}
\subsection{概率的连续性}

\section{条件概率}
\subsection{条件概率的定义}
\subsection{乘法公式}
\subsection{全概率公式}
\subsection{贝叶斯公式}

\section{独立性}
\subsection{两个事件的独立性}
\subsection{多个事件的相互独立性}
\subsection{试验的独立性}